% Hlavicka pro protokoly z fyzikalniho praktika.
% Verze pro: LaTeX
% Verze hlavicky: 22. 2. 2007
% Autor: Ustav fyziky kondenzovanych latek
% Ke stazeni: www.physics.muni.cz/ufkl/Vyuka/
% Licence: volne k pouziti, nejlepe k vcasnemu odevzdani protokolu z Vaseho mereni.

\documentclass[a4paper,11pt]{article}

% Kodovani (cestiny) v dokumentu: cp1250
% \usepackage[cp1250]{inputenc}	% Omezena stredoevropska kodova stranka, pouze MSW.
\usepackage[utf8]{inputenc}	% Doporucujeme pouzivat UTF-8 (unicode).

%%% Nemente:
\usepackage[margin=2cm]{geometry}
\newtoks\jmenopraktika \newtoks\jmeno \newtoks\datum
\newtoks\obor \newtoks\skupina \newtoks\rocnik \newtoks\semestr
\newtoks\cisloulohy \newtoks\jmenoulohy
\newtoks\tlak \newtoks\teplota \newtoks\vlhkost
%%% Nemente - konec.


%%%%%%%%%%% Doplnte pozadovane polozky:

\jmenopraktika={Fyzikální praktikum 1}  % nahradte jmenem vaseho predmetu
\jmeno={Milan Suk}            % nahradte jmenem mericiho
\datum={2. března 2017}        % nahradte datem mereni ulohy
\obor={F}                     % nahradte zkratkou vami studovaneho oboru
\skupina={ČT 8:00}            % nahradte dobou vyuky vasi seminarni skupiny
\rocnik={I}                  % nahradte rocnikem, ve kterem studujete
\semestr={II}                 % nahradte semestrem, ve kterem studujete

\cisloulohy={1}               % nahradte cislem merene ulohy
\jmenoulohy={Měření hustosty válečku} % nahradte jmenem merene ulohy

\tlak={97,9}                   % nahradte tlakem pri mereni (v hPa)
\teplota={21,4}               % nahradte teplotou pri mereni (ve stupnich Celsia)
\vlhkost={40}               % nahradte vlhkosti vzduchu pri mereni (v %)

%%%%%%%%%%% Konec pozadovanych polozek.


%%%%%%%%%%% Uzitecne balicky:
\usepackage[czech]{babel}
\usepackage{graphicx}
\usepackage{amsmath}
\usepackage{xspace}
\usepackage{url}
\usepackage{indentfirst}

%%%%%% Zamezeni parchantu:
\widowpenalty 10000 \clubpenalty 10000 \displaywidowpenalty 10000
%%%%%% Parametry pro moznost vsazeni vetsiho poctu obrazku na stranku
\setcounter{topnumber}{3}	  % max. pocet floatu nahore (specifikace t)
\setcounter{bottomnumber}{3}	  % max. pocet floatu dole (specifikace b)
\setcounter{totalnumber}{6}	  % max. pocet floatu na strance celkem
\renewcommand\topfraction{0.9}	  % max podil stranky pro floaty nahore
\renewcommand\bottomfraction{0.9} % max podil stranky pro floaty dole
\renewcommand\textfraction{0.1}	  % min podil stranky, ktery musi obsahovat text
\intextsep=8mm \textfloatsep=8mm  %\intextsep pro ulozeni [h] floatu a \textfloatsep pro [b] or [t]

% Tecky za cisly sekci:
\renewcommand{\thesection}{\arabic{section}.}
\renewcommand{\thesubsection}{\thesection\arabic{subsection}.}
% Jednopismenna mezera mezi cislem a nazvem kapitoly:
\makeatletter \def\@seccntformat#1{\csname the#1\endcsname\hspace{1ex}} \makeatother


%%%%%%%%%%%%%%%%%%%%%%%%%%%%%%%%%%%%%%%%%%%%%%%%%%%%%%%%%%%%%%%%%%%%%%%%%%%%%%%
%%%%%%%%%%%%%%%%%%%%%%%%%%%%%%%%%%%%%%%%%%%%%%%%%%%%%%%%%%%%%%%%%%%%%%%%%%%%%%%
% Zacatek dokumentu
%%%%%%%%%%%%%%%%%%%%%%%%%%%%%%%%%%%%%%%%%%%%%%%%%%%%%%%%%%%%%%%%%%%%%%%%%%%%%%%
%%%%%%%%%%%%%%%%%%%%%%%%%%%%%%%%%%%%%%%%%%%%%%%%%%%%%%%%%%%%%%%%%%%%%%%%%%%%%%%

\begin{document}

%%%%%%%%%%%%%%%%%%%%%%%%%%%%%%%%%%%%%%%%%%%%%%%%%%%%%%%%%%%%%%%%%%%%%%%%%%%%%%%
% Nemente:
%%%%%%%%%%%%%%%%%%%%%%%%%%%%%%%%%%%%%%%%%%%%%%%%%%%%%%%%%%%%%%%%%%%%%%%%%%%%%%%
\thispagestyle{empty}

{
\begin{center}
\sf 
{\Large Ústav fyzikální elektroniky Přírodovědecké fakulty Masarykovy univerzity} \\
\bigskip
{\huge \bfseries FYZIKÁLNÍ PRAKTIKUM} \\
\bigskip
{\Large \the\jmenopraktika}
\end{center}

\bigskip

\sf
\noindent
\setlength{\arrayrulewidth}{1pt}
\begin{tabular*}{\textwidth}{@{\extracolsep{\fill}} l l}
\large {\bfseries Zpracoval:}  \the\jmeno & \large  {\bfseries Naměřeno:} \the\datum\\[2mm]
\large  {\bfseries Obor:} \the\obor  \hspace{40mm}  {\bfseries Skupina:} \the\skupina %
%{\bfseries Ročník:} \the\rocnik \hspace{5mm} {\bfseries Semestr:} \the\semestr  
&\large {\bfseries Testováno:}\\
\\
\hline
\end{tabular*}
}

\bigskip

{
\sf
\noindent \begin{tabular}{p{3cm} p{0.6\textwidth}}
\Large  Úloha č. {\bfseries \the\cisloulohy:} \par
\smallskip
$T=\the\teplota$~$^\circ$C \par
$p=\the\tlak$~hPa \par
$\varphi=\the\vlhkost$~\%
&\Large \bfseries \the\jmenoulohy  \\[2mm]
\end{tabular}
}

\vskip1cm

%%%%%%%%%%%%%%%%%%%%%%%%%%%%%%%%%%%%%%%%%%%%%%%%%%%%%%%%%%%%%%%%%%%%%%%%%%%%%%%
% konec Nemente.
%%%%%%%%%%%%%%%%%%%%%%%%%%%%%%%%%%%%%%%%%%%%%%%%%%%%%%%%%%%%%%%%%%%%%%%%%%%%%%%

%%%%%%%%%%%%%%%%%%%%%%%%%%%%%%%%%%%%%%%%%%%%%%%%%%%%%%%%%%%%%%%%%%%%%%%%%%%%%%%
%%%%%%%%%%%%%%%%%%%%%%%%%%%%%%%%%%%%%%%%%%%%%%%%%%%%%%%%%%%%%%%%%%%%%%%%%%%%%%%
% Zacatek textu vlastniho protokolu
%%%%%%%%%%%%%%%%%%%%%%%%%%%%%%%%%%%%%%%%%%%%%%%%%%%%%%%%%%%%%%%%%%%%%%%%%%%%%%%
%%%%%%%%%%%%%%%%%%%%%%%%%%%%%%%%%%%%%%%%%%%%%%%%%%%%%%%%%%%%%%%%%%%%%%%%%%%%%%%


\section{Úvod}

    \paragraph{} Cílem toho měření je zjistit hustotu válečku s válcovým výřezem. 
    Hustota má být vypočtena pomocí měření jeho výšky, vnitřního a vnějšího průměru, a
    jeho hmotnosti.

    \paragraph{} Objem válečku určím pomocí změřené výšky $h$, vnitřního poloměru $r$ a
    vnějšího poloměr $R$ pomocí rovnice

    $$V = h \pi (R^{2} - r^{2})$$

    \paragraph{} Pak se změřenou hmotností $m$ lze určit hustota válečku jako

    $$\rho = \frac{m}{h \pi (R^{2} - r^{2})}$$

    \subsection{Zpracování chyb měření}

        \paragraph{} Nejistotu měření určím ze \textit{Zákona šíření nejistoty} jako

        $$u(\rho) = \sqrt{
          \left(\frac{\partial \rho}{\partial m}\right)^{2} \cdot u(m)^{2}
        + \left(\frac{\partial \rho}{\partial r}\right)^{2} \cdot u(r)^{2} 
        + \left(\frac{\partial \rho}{\partial R}\right)^{2} \cdot u(R)^{2}
        + \left(\frac{\partial \rho}{\partial h}\right)^{2} \cdot u(h)^{2}
        }$$

        konkrétně potom

        $$u(\rho) = \sqrt{
              \left(\frac{u(m)}{h \pi \left(R^{2} - r^{2}\right)}\right)^{2}
            + \left(\frac{2mr \cdot u(R)}{h \pi \left(R^{2} - r^{2}\right)^{2}}\right)^{2}
            + \left(\frac{2mR \cdot u(r)}{h \pi \left(R^{2} - r^{2}\right)^{2}}\right)^{2}
            + \left(\frac{m \cdot u(h)}{h^{2} \pi \left(R^{2} - r^{2}\right)}\right)^{2}
        }$$

\section{Postup měření}

    \paragraph{} Pomocí posuvného měřidla jsme nejdříve změřili vnitřní průměr $2r$ a vnější průměr 
    válečku $2R$. Poté jsme pomocí mikrometru změřili výšku válečku $h$ a nakonec pomocí laboratorních vah jsme 
    určili hmotnost válečku $m$.

\section{Výsledky}

    \subsection{Měření průměrů válce}

    \begin{center}
        \begin{tabular}{ | l | l || l | l | }
            \hline
            $2R$ $[cm]$ & $R$ $[cm]$ & $2r$ $[cm]$ & $r$ $[cm]$ \\ \hline
            3.992 & 1.996 & 1.008 & 0.504 \\ \hline
            3.992 & 1.996 & 1.008 & 0.504 \\ \hline
            3.990 & 1.995 & 0.996 & 0.498 \\ \hline
            3.992 & 1.996 & 1.006 & 0.502 \\ \hline
            3.990 & 1.995 & 1.008 & 0.504 \\ \hline
            3.992 & 1.996 & 1.002 & 0.501 \\ \hline
            3.992 & 1.996 & 1.006 & 0.503 \\ \hline
            3.992 & 1.996 & 1.000 & 0.500 \\ \hline
            3.992 & 1.996 & 1.008 & 0.504 \\ \hline
            3.994 & 1.997 & 1.010 & 0.505 \\
            \hline
        \end{tabular}
    \end{center}

    \subsection{Měření výšky válce}

    \begin{center}
        \begin{tabular}{ | l | }
            \hline
            $h$ $[cm]$ \\ \hline
            1.534 \\ \hline
            1.530 \\ \hline
            1.528 \\ \hline
            1.524 \\ \hline
            1.552 \\ \hline
            1.546 \\ \hline
            1.548 \\ \hline
            1.548 \\ \hline
            1.554 \\ \hline
            1.550 \\
            \hline
        \end{tabular}
    \end{center}

    $$\overline{m} = 159.096g$$

    \paragraph{} Průměrné hodnoty naměřených veličin a jejich nejistoty jsou následující:

    $$m = (159.096 \pm 0.001)g$$
    $$r = (0.503 \pm 0.001)cm$$
    $$R = (1.996 \pm 0.001)cm$$
    $$h = (1.541 \pm 0.004)cm$$

    \paragraph{} Po dosazení je výsledná hustota

    $$\rho = \left(8.81 \pm 0.02 \right) g \cdot cm^{-3}$$

\section{Zhodnocení měření, závěr}

    \paragraph{} Podle zjištěné hustoty byl neznámý materiál pravděpodobně \textbf{mosaz}, jehož tabulková
    hodnota hustoty se pohybuje v rozmezí $8.4 - 8.7$ $g \cdot cm^{-3}$. Odchylka od této hodnoty bude 
    pravděpodobně způsobena nedokonalým tvarem, o kterém se předpokládalo, že se jedná o válec. Měření by
    bylo možné zlepšit přímým měřením objemu, ponořením do kapaliny a měřením jejího objemu v odměrném válci.

\end{document}

\end{document}

